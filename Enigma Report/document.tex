\documentclass[a4paper,12pt]{article}
\usepackage[top=2cm,bottom=2cm,left=2cm,right=2cm]{geometry}
\usepackage[utf8]{inputenc}
\usepackage{graphicx}
\usepackage[fontset=macnew]{ctex}
\usepackage{verbatim}
\usepackage{enumitem}
\usepackage{floatrow}
\usepackage{stmaryrd}
\usepackage{amsmath,amsthm,amssymb}
\usepackage{multicol}

\setlist{noitemsep}

\title{Enigma破译\ 实验报告}
\author{陈庆之 2021011819}

\begin{document}
	
	\maketitle
	
	\section{算法原理}
	\subsection{Enigma原理}
	
	\subsection{Rejewski的方法}
	
	\subsection{Turing的方法}
	
	\section{实际攻击样例}
	
	\subsection{Rejewski的方法}
	
	
	
	\subsection{Turing的方法}
	
	\section{代码结构}
	
	本次实验的代码部分包括以下文件:
	\begin{enumerate}
		\item \texttt{enigma.py}: 实现了支持选择转子顺序、插线板、转子设置、初始值设置的Enigma I密码机。密码机会存储最开始的设置,并支持\texttt{reset()}方法。密码机的转子数量、可用转子排列等可以被简单地扩展。
		
		\item \texttt{rejewski.py}: 实现了Rejewski的破解方法。其中的\texttt{make\_catalogue()}可以在当前目录下生成\texttt{catalogue.json}(相当于波兰人\textbf{建立目录}的过程),之后使用\texttt{decypher(...)}方法可以对特定的重复密钥序列进行破解。
		
		\item \texttt{turing.py}: 实现了Turing的破解方法。使用其中的\texttt{decypher(...)}方法并传入已发现的环,方法会返回所有可能的转子序列和初始位置。
	\end{enumerate}
	
	本次实验没有使用第三方包,只需\texttt{python3}即可。
	
\end{document}